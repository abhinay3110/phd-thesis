% --- French package
%\usepackage[french,greek]{babel}

% --- Front page Font
\newenvironment{frontfont}{\fontfamily{phv}\selectfont}{\par}
\DeclareTextFontCommand{\textFF}{\frontfont}

% --- Space between paragraphs/enums/items...
\setlength{\parskip}{0.5em}
\raggedbottom
\usepackage{etaremune}
\usepackage[inline]{enumitem}
\setlist[enumerate,itemize,description]{topsep=0em}

% --- Algorithms packages
\usepackage[algochapter,linesnumbered,commentsnumbered,inoutnumbered]{algorithm2e}
\SetKwComment{Comment}{$\triangleright$\ }{}

% --- Maths packages
\usepackage{amsmath}
% https://tex.stackexchange.com/a/12561/97964
\allowdisplaybreaks
\usepackage{amsthm}
\usepackage{amssymb}
\usepackage{bbm}
\usepackage{bm}
\usepackage{mathrsfs}
\usepackage{mathtools}
\usepackage{amsfonts}
\usepackage{empheq}
\usepackage{blkarray}
\usepackage{stmaryrd}
\usepackage{setspace}
\usepackage{stmaryrd}
\usepackage{cases}
\usepackage{stackengine}
\usepackage{derivative}

% --- Figures and Captions
\usepackage{epstopdf}
\usepackage[small,bf,labelsep=endash,tableposition=bottom]{caption}
\usepackage{graphicx}
\DeclareGraphicsExtensions{.jpg,.pdf,.mps,.eps,.png}
\usepackage[list=true]{subcaption}
\usepackage{cleveref}
\usepackage{placeins}

% --- Figures Tikz
\usepackage{tikz}
\usetikzlibrary{positioning}
\usetikzlibrary{fit}
\usetikzlibrary{arrows}
\usetikzlibrary{shapes.geometric}
\usetikzlibrary{shapes.symbols}
\usetikzlibrary{backgrounds}
\usetikzlibrary{chains}
\usetikzlibrary{automata}
\usetikzlibrary{intersections}
\usetikzlibrary{decorations.pathreplacing}
\usepackage{tikzsymbols}

% --- Color definition
\usepackage{xcolor}
\definecolor{Bleu}{RGB}{0,0,204}           % Define a new color rgb(0,0,204)
\definecolor{darkblue}{RGB}{0,0,126}       % Define a new color rgb(0,0,126)
\definecolor{Violet}{RGB}{102,0,204}       % Define a new color rgb(102,0,204)
\definecolor{deeppurple}{RGB}{102,0,204}   % Define a new color rgb(102,0,204)
\definecolor{darkgreen}{RGB}{0,100,0}      % Define a new color rgb(0,100,0)
\definecolor{lightgreen}{RGB}{185,220,87}      % Define a new color rgb(185,220,87)
\definecolor{gold}{RGB}{255,184,0}         % Define a new color rgb(255,184,0)
\definecolor{deepgold}{RGB}{205,105,0}     % Define a new color rgb(205,105,0)
\definecolor{Rouge}{RGB}{204,0,0}          % Define a new color rgb(204,0,0)
\definecolor{darkred}{RGB}{174,0,0}        % Define a new color rgb(174,0,0)
\definecolor{Highlight}{RGB}{251,0,0}      % Define a new color rgb(251,0,0)
\definecolor{gold}{RGB}{255,184,0}         % Define a new color rgb(255,184,0)

% --- Tables
\usepackage{booktabs}
\usepackage{multirow}
\usepackage{multicol}
\usepackage{rotating}
\usepackage{array}
\usepackage{multirow}
\usepackage{arydshln}
\usepackage{hhline}
\usepackage{makecell}
\usepackage[flushleft]{threeparttable}

% --- SI Units
\usepackage[binary-units,detect-all]{siunitx}

% --- Glossaries
\usepackage[acronym,toc,symbols,style=alttreegroup]{glossaries-extra}
\glssetwidest{app this width}
\makeglossaries
\setabbreviationstyle{long-short}
%\glssetcategoryattribute{general}{textformat}{textsf}
\glssetcategoryattribute{locutions}{textformat}{emph}
\glssetcategoryattribute{algorithm}{textformat}{texttt}
\glsxtrabbreviationfont
\GlsXtrEnableInitialTagging{acronym,abbreviation}{\itag}
\GlsXtrEnableLinkCounting{abbreviation} % Link first occurence
\GlsXtrEnableLinkCounting[chapter]{acronym,general,symbol}  % Link first occurence in each chapter
\renewcommand*{\glslinkpresetkeys}{% % disable hyperlink if link count is greater than 1:
	\ifnum\GlsXtrLinkCounterValue{\glslabel}>1
	\setkeys{glslink}{hyper=false}%
	\fi
}
\glssetcategoryattribute{symbol}{indexonlyfirst}{true}  % Index first occurence of symbols only
\glssetcategoryattribute{general}{nohyper}{true}  % Disable hyperref for general glossary
\usepackage{makeidx} % Required to make an index
%\makeindex
% --------------------------------
% matches Abbreviations        > A
%\nomenclature[A]{wrt}{\textit{with respect to}}
%\nomenclature[A]{wlog}{\textit{with loss of generality}}
%\nomenclature[A]{i.i.d.}{\textit{identically and independently distributed} (variables, observations or samples)}
%\nomenclature[A]{ADAS}{\textit{Advanced Driver-Assistance Systems}}
%\nomenclature[A]{ACC}{\textit{Adaptive Cruise Control}}
%\nomenclature[A]{LKA}{\textit{Lane-Keeping Assist}}
% Algorithms or tools
%\nomenclature[A]{ML}{\textit{Machine Learning}}
%\nomenclature[A]{RL}{\textit{Reinforcement Learning}}
%\nomenclature[A]{MDP}{\textit{Markov Decision Process}}
%\nomenclature[A]{POMDP}{\textit{Partially Observable Markov Decision Process}}
%\nomenclature[A]{LQR}{\textit{Linear-Quadratic Regulator}}
%\nomenclature[A]{LQG}{\textit{Linear-Quadratic-Gaussian}}
% --------------------------------
% matches Greek Symbols        > G
\nomenclature[G]{$\gamma$}{Usually denotes a discount factor}

% --------------------------------
% matches Roman Symbols        > L
\nomenclature[L]{$t$}{Time step, $t\in[T]$}

% --------------------------------
% matches mathematical symbols > O
\nomenclature[O]{$\Pr$}{Probability measure under a probabilistic model}
\nomenclature[O]{$\E$}{Expectation under a probabilistic model}

% --------------------------------
% matches Subscripts           > I
\nomenclature[I]{$x_t$}{Usually denotes a variable depending on time, for $t\in[T]$}

% --------------------------------
% matches Superscripts         > E
\nomenclature[E]{$y^j$}{Usually denotes ...}

\chapter*{List of Acronyms}
\addcontentsline{toc}{chapter}{List of Acronyms}
\begin{acronym}[TDMA]
\acro{AD}{Autonomous Driving}
\acro{ADAS}{Advanced Driver-Assistance Systems}
\acro{ACC}{Adaptive Cruise Control}
\acro{LKA}{Lane-Keeping Assist}
\acro{AEB}{Autonomous Emergency Braking}
\acro{AES}{Autonomous Emergency Steering}
\acro{RL}{Reinforcement Learning}
\acro{MDP}{Markov Decision Process}
\acro{ML}{Machine Learning}
\acro{PAC}{Probably Approximately Correct}
\acro{MLE}{Maximum Likelihood Estimation}
\acro{MCTS}{Monte-Carlo Tree Search}
\acro{PRM}{Probabilistic Roadmap}
\acro{RRT}{Rapidly-exploring Random Trees}
\end{acronym}



\definecolor{symbolcolor}{rgb}{0.5,0.,0}
\newcommand{\var}[1]{\ensuremath{#1}}

% TODO: Aliases, to be removed
\newcommand{\idx}[1]{#1\index{#1}}
\newcommand{\discountfactor}{\discount}
\newcommand{\augmentedreward}{\oR}
\newcommand{\reward}{\R}
\newcommand{\transition}{\Ps}


% --- Mathematical notations: key A

\newglossaryentry{real}{name={$\mathbb{R}$},description={set of real numbers}, type=symbols, category=symbol, sort={a}}
\newglossaryentry{natural}{name={$\mathbb{N}$},description={set of integers}, type=symbols, category=symbol, sort={a}}
\newglossaryentry{realplus}{name={$\mathbb{R}_{+}$},description={set of positive reals $\{\tau\in\mathbb{R}:\tau\ge0\}$}, type=symbols, category=symbol, sort={a}}
\newglossaryentry{naturalplus}{name={$\mathbb{N}_{+}$},description={set of positive integers $\mathbb{N}\cap\mathbb{R}_{+}$}, type=symbols, category=symbol, nonumberlist, sort={a}}
\newglossaryentry{euclideannorm}{name={$|x|$},description={Euclidean norm for a vector $x\in\mathbb{R}^{n}$}, type=symbols, category=symbol, nonumberlist, sort={a}}
\newglossaryentry{infinitynorm}{name={$\|u\|_{[t_{0},t_{1}]}$},description={$L_{\infty}$ norm on $[t_{0},t_{1})$ of a measurable and locally essentially bounded input $u:\mathbb{R}_{+}\to\mathbb{R}$}, type=symbols, category=symbol, nonumberlist, sort={a}}
\newglossaryentry{infinitynorm2}{name={$\|u\|$},description={$L_{\infty}$ norm $||u||_{[t_{0},t_{1}]}$ with $t_1=+\infty$}, type=symbols, category=symbol, nonumberlist, sort={a}}
\newglossaryentry{bounded}{name={$\mathcal{L}_{\infty}$},description={the set of all inputs $u$ with the property $||u||<\infty$}, type=symbols, category=symbol, nonumberlist, sort={a}}
\newglossaryentry{pos}{name={$z^+$},description={positive part $\max(z,0)$}, type=symbols, category=symbol, sort={a}}
\newglossaryentry{neg}{name={$z^-$},description={negative part $z^- = z^+-z$}, type=symbols, category=symbol, sort={a}}
\newglossaryentry{abs}{name={$|z|$},description={absolute value $|z| = z^++z^-$}, type=symbols, category=symbol, sort={a}}
\newglossaryentry{range}{name={$[n]$},description={range of integers $\{1,\dots, n\}$}, type=symbols, category=symbol, sort={a}}
\newglossaryentry{identity}{name={$I_{n}$},description={the identity matrix with dimension $n\times n$}, type=symbols, category=symbol, sort={a}}
\newglossaryentry{ones}{name={$E_{n\times m},\;E_{p}$},description={the matrices with all elements equal 1 with dimensions $n\times m$ and $p\times1$, respectively}, type=symbols, category=symbol, sort={a}}
\newglossaryentry{basis}{name={$e_{i}$},description={normal basis vectors $[0\dots0\;1\;\dots0]\tr$ in $\mathbb{R}^{n}$ for $i=\overline{1,n}$, where $1$ appears in the $i^{\text{th}}$ position}, type=symbols, category=symbol, sort={a}}
\newglossaryentry{spectrum}{name={$\lambda(A)$},description={the vector of eigenvalues of a matrix $A\in\mathbb{R}^{n\times n}$}, type=symbols, category=symbol, sort={a}}
\newglossaryentry{maximumnorm}{name={$\|A\|_{\max}$},description={the elementwise maximum norm $\|A\|_{max}=\max_{i\in[n],j\in[n]}|A_{i,j}|$, it is not sub-multiplicative}, type=symbols, category=symbol, sort={a}}
\newglossaryentry{l2norm}{name={$\|A\|_{2}$}, description={the induced $L_{2}$ matrix norm $\max_{i\in[n]}\lambda_{i}(A^{\top} A)$}, type=symbols, category=symbol, sort={a}}
\newglossaryentry{vleq}{name={$x_{1}\le x_{2}$}, description={for two matrices $A_{1},A_{2}\in\Real^{n\times n}$, (including vectors), the relation $A_{1}\le A_{2}$ is understood elementwise}, type=symbols, category=symbol, sort={a}}
\newglossaryentry{posdef}{name={$P\prec0$ ($P\succ0$)}, description={a symmetric matrix $P\in\Real^{n\times n}$ is negative (positive) definite}, type=symbols, category=symbol, sort={a}}
\newglossaryentry{bigo}{name={$o(\cdot), \mathcal{O}(\cdot), \Omega(\cdot)$}, description={Landau notations for positive functions: $f(x)=o(g(x)$ means that $g(x)\neq0$ and $f(x)/g(x)\to 0$ for $x\to\infty$, $f(x)=\mathcal{O}(g(x))$ means that there exists $x_0,K>0$ such that $f(x)\leq K g(x)$ from $x\geq x_0$, and $f(x)=\Omega(g(x))$ means $g(x)=\mathcal{O}(f(x))$}, type=symbols, category=symbol, sort={a}}


% --- Probabilities: key AP

\newglossaryentry{expectedvalue}{name={$\mathbb{E}$},description={expectation under a probabilistic model},type=symbols, category=symbol, sort={ap}}
\newglossaryentry{variance}{name={$\mathbb{V}$},description={variance under a probabilistic model},type=symbols, category=symbol, sort={ap}}

\newglossaryentry{cM}{name={$\mathcal{M}(\cX)$}, text={\var{\mathcal{M}}},description={set of probability measures on a measurable space $\cX$},type=symbols, category=symbol, sort={ap}}
\newcommand{\cM}{\gls{cM}}

\newglossaryentry{uniform}{name={$\mathcal{U}(\cX)$}, text={\var{\mathcal{U}}}, description={uniform distribution on a measurable space $\cX$},type=symbols, sort={ap}}

\newglossaryentry{dirac}{name={\var{\delta}},description={Dirac distribution},type=symbols, category=symbol, sort={ap}}
\newcommand{\dirac}{\gls{dirac}}

\newglossaryentry{normald}{name={\var{\mathcal{N}}},description={Normal distribution},type=symbols, category=symbol, sort={ap}}
\newcommand{\normal}{\gls{normald}}

\newglossaryentry{binomiald}{name={\var{\mathcal{B}}},description={Binomial distribution},type=symbols, category=symbol, sort={ap}}
\newcommand{\binomial}{\gls{binomiald}}

\newglossaryentry{transpose}{name={$M^\top$}, text={\var{\top}},description={transpose of a matrix $M$},type=symbols, category=symbol, sort={ap}}
\newcommand{\transpose}{\gls{transpose}}

% --- MDP: key B MDP

\newglossaryentry{cS}{name={\var{\mathcal{S}}},description={set of states $s\in\mathcal{S}$},type=symbols, category=symbol, sort={bmdpa}}
\newcommand{\cS}{\gls{cS}}

\newglossaryentry{cA}{name={\var{\mathcal{A}}},description={set of actions $a\in\mathcal{A}$},type=symbols, category=symbol, sort={bmdpb}}
\newcommand{\cA}{\gls{cA}}

\newglossaryentry{reward}{name={$R(s,a)$}, text={\var{R}},description={reward function $R:s,a\rightarrow R(s,a) \in [0, 1]$},type=symbols, category=symbol, sort={bmdpc}}
\newcommand{\R}{\gls{reward}}

\newglossaryentry{transition}{name={$P\left(s'\mid s,a\right)$},text={\var{P}},description={transition distribution $s'\sim P\parentheses{s'\mid s,a}$},type=symbols, category=symbol, sort={bmdpd}}
\newcommand{\Ps}{\gls{transition}}

\newglossaryentry{discountfactor}{name={\var{\gamma}},description={discount factor in $[0, 1)$},type=symbols, category=symbol, sort={bmdpe}}
\newcommand{\discount}{{\gls{discountfactor}}}

\newglossaryentry{policy}{name={\var{\pi}},description={policy},type=symbols, category=symbol, sort={bmdpf}}
\newcommand{\policy}{{\gls{policy}}}

\newglossaryentry{optimalpolicy}{name={\var{\pi^\star}},description={optimal policy},type=symbols, category=symbol, sort={bmdpg}}
\newcommand{\optimalpolicy}{\gls{optimalpolicy}}

\newglossaryentry{return}{name={\var{G}},description={discounted return for the reward signal},type=symbols, category=symbol, sort={bmdph}}
\newcommand{\return}{\gls{return}}

\newglossaryentry{V}{name={\var{V}},description={state value function (${}^\star$ for optimal value, ${}^\pi$ for policy value)},type=symbols, category=symbol, sort={bmdpi}}
\newcommand{\V}{\gls{V}}

\newglossaryentry{Q}{name={\var{Q}},description={state-action value function (${}^\star$ for optimal value, ${}^\pi$ for policy value)},type=symbols, category=symbol, sort={bmdpj}}
\newcommand{\Q}{\gls{Q}}

\newglossaryentry{bo}{name={\var{\mathcal{T}}},description={Bellman operator (${}^\star$ for optimality, ${}^\pi$ for evaluation)},type=symbols, category=symbol, sort={bmdpk}}
\newcommand{\bo}{\gls{bo}}

\newglossaryentry{regret}{name={\var{r_n}},description={simple regret of an algorithm},type=symbols, category=symbol, sort={bmdpl}}
\newcommand{\regret}{\gls{regret}}

% --- Budgeted RL: key E BRL

\newglossaryentry{ocS}{name={\var{\overline{\mathcal{S}}}},description={set of augmented states},type=symbols, category=symbol, sort={ebrla}}
\newcommand{\ocS}{\gls{ocS}}

\newglossaryentry{ocA}{name={\var{\overline{\mathcal{A}}}},description={set of augmented actions},type=symbols, category=symbol, sort={ebrlb}}
\newcommand{\ocA}{\gls{ocA}}

\newglossaryentry{cBB}{name={\var{\mathscr{B}}},description={set of admissible budgets},type=symbols, category=symbol, sort={ebrlc}}
\newcommand{\budgetspace}{\gls{cBB}}

\newglossaryentry{piii}{name={\var{\overline{\Pi}}},description={set of budgeted policies},type=symbols, category=symbol, sort={ebrld}}
\newcommand{\policies}{\gls{piii}}

\newglossaryentry{constraintx}{name={\var{C}},description={cost function},type=symbols, category=symbol, sort={ebrle}}
\newcommand{\constraint}{\gls{constraintx}}

\newglossaryentry{budget}{name={\var{\beta}},description={budget},type=symbols, category=symbol, sort={ebrlf}}
\newcommand{\budget}{\gls{budget}}

\newglossaryentry{budgetaction}{name={\var{\beta_a}},description={budget allocated to an action},type=symbols, category=symbol, sort={ebrlg}}
\newcommand{\budgetaction}{\gls{budgetaction}}

\newglossaryentry{augmentedtransition}{name={\var{\overline{P}}},description={augmented transition function},type=symbols, category=symbol, sort={ebrlh}}
\newcommand{\augmentedtransition}{\gls{augmentedtransition}}

\newglossaryentry{oR}{name={\var{\overline{R}}},description={augmented reward function},type=symbols,category=symbol, sort={ebrli}}
\newcommand{\oR}{\gls{oR}}

\newglossaryentry{budgetedpolicy}{name={\var{\overline{\pi}}},description={budgeted policy},type=symbols, category=symbol, sort={ebrlj}}
\newcommand{\budgetedpolicy}{{\gls{budgetedpolicy}}}

\newglossaryentry{optimalbudgetedpolicy}{name={\var{\overline{\pi}^\star}},description={optimal budgeted policy},type=symbols, category=symbol, sort={ebrlk}}
\newcommand{\optimalbudgetedpolicy}{\gls{optimalbudgetedpolicy}}

\newglossaryentry{constraintreturn}{name={\var{G_c}},description={discounted return for the cost signal},type=symbols, category=symbol, sort={ebrll}}
\newcommand{\constraintreturn}{\gls{constraintreturn}}

\newglossaryentry{augmentedreturn}{name={\var{\overline{G}}},description={augmented return},type=symbols, category=symbol, sort={ebrlm}}
\newcommand{\augmentedreturn}{\gls{augmentedreturn}}

\newglossaryentry{oV}{name={\var{\overline{V}}},description={augmented value function},type=symbols, category=symbol, sort={ebrln}}
\newcommand{\oV}{\gls{oV}}

\newglossaryentry{Vr}{name={\var{V_r}},description={reward value function},type=symbols, category=symbol, sort={ebrln}}
\newcommand{\Vr}{\gls{Vr}}

\newglossaryentry{Vc}{name={\var{V_c}},description={cost value function},type=symbols, category=symbol, sort={ebrln}}
\newcommand{\Vc}{\gls{Vc}}

\newglossaryentry{oQ}{name={\var{\overline{Q}}},description={augmented Q-function},type=symbols, category=symbol, sort={ebrlo}}
\newcommand{\oQ}{\gls{oQ}}

\newglossaryentry{Qr}{name={\var{Q_r}},description={reward Q-function},type=symbols, category=symbol, sort={ebrlo}}
\newcommand{\Qr}{\gls{Qr}}

\newglossaryentry{Qc}{name={\var{Q_c}},description={Cost Q-function},type=symbols, category=symbol, sort={ebrlo}}
\newcommand{\Qc}{\gls{Qc}}

\newglossaryentry{abo}{name={\var{\overline{\mathcal{T}}}},description={augmented Bellman operator (${}^\star$ for optimality, ${}^\pi$ for evaluation)},type=symbols, category=symbol, sort={ebrlp}}
\newcommand{\abo}{{\gls{abo}}}



% --- Linear Systems: Key = Z CTRL

\newglossaryentry{safestates}{name={\var{\mathbb{X}}},description={constraint set for safe states $x(t)\in\mathbb{X}\subset\mathbb{R}^p$},type=symbols, category=symbol, sort={zctrla}}
\newcommand{\safestates}{{\gls{safestates}}}

\newglossaryentry{safecontrols}{name={\var{\mathbb{U}}},description={constraint set for safe controls $u(t)\in\gls{safecontrols}\subset\mathbb{R}^q$},type=symbols, category=symbol, sort={zctrlb}}
\newcommand{\safecontrols}{{\gls{safecontrols}}}

\newglossaryentry{timestep}{name={\var{\mathrm{d} t}},description={time step at which \gls{MPC} controls are applied},type=symbols, category=symbol, sort={zctrlc}}
\newcommand{\timestep}{\gls{timestep}}

\newglossaryentry{features}{name={\var{\phi}},description={features for a parametrized},type=symbols, category=symbol, sort={zctrld}}
\newcommand{\features}{\gls{features}}

\newglossaryentry{params}{name={\var{\theta}},description={parameters $\params\in\mathbb{R}^d$ for a model},type=symbols, category=symbol, sort={zctrle}}
\newcommand{\params}{{\gls{params}}}

\newglossaryentry{structureddynamics}{name={\var{A(\params)}},description={structured state matrix $A(\params)\in\mathbb{R}^{p\times p}$, depending on unknown parameters $\params$},type=symbols, category=symbol, sort={zctrlf}}
\newcommand{\structureddynamics}{{\gls{structureddynamics}}}

\newglossaryentry{confidence}{name={\var{\delta}},description={confidence level for statistical estimates, in $(0, 1]$},type=symbols, category=symbol, sort={zctrlg}}
\newcommand{\confidence}{{\gls{confidence}}}

\newglossaryentry{confidenceset}{name={\var{\mathcal{C}_{[N],\delta}}},description={high-confidence set for the estimation of $\params$, such that $\mathbb{P}(\params\in\confidenceset) \geq 1-\confidence$.},type=symbols, category=symbol, sort={zctrlh}}
\newcommand{\confidenceset}{\gls{confidenceset}}

\newglossaryentry{Tt}{name={\var{N}},description={number of transition samples},type=symbols, category=symbol, sort={zctrli}}
\newcommand{\T}{\gls{Tt}}

\newglossaryentry{maxit}{name={\var{K}},description={number of planning iterations},type=symbols, category=symbol, sort={zctrlj}}
\newcommand{\maxiteration}{\gls{maxit}}



% --- Others, unsorted

\newglossaryentry{cD}{name={\var{\mathcal{D}}},description={dataset},type=symbols, category=symbol}
\newcommand{\cD}{\gls{cD}}




% --- Bibliography
% Add `custombib' in the document class option to use this section

	
\ifuseCustomBib

	\usepackage[backend=biber, style=authoryear,
		sorting=nyt, natbib=true,doi=false,isbn=false,url=false,eprint=false]{biblatex}
	\setlength\bibitemsep{0.5\baselineskip}

	\DeclareFieldFormat[article, inbook]{title}{#1} 
  \renewcommand*{\intitlepunct}{\addspace\nopunct}
	\renewbibmacro{in:}{%
		\ifentrytype{article}
		{}
		{\printtext{\bibstring{in}\intitlepunct}}}
		
%	\usepackage[square,sort,authoryear]{natbib}
%	\usepackage[style=apa, backend=biber,natbib=true]{biblatex}
    %\usepackage[sort,numbers]{natbib}
	%\usepackage[backend=bibtex,style=alphabetic,natbib=true]{biblatex}
	%\renewbibmacro{in:}{}
	%\DefineBibliographyExtras{french}{\restorecommand\mkbibnamelast}
	%\bibliography{all-phd-thesis}
\fi
% changes the default name `Bibliography` -> `References'
\renewcommand{\bibname}{List of References}
\renewcommand{\contentsname}{Table of Contents}
\usepackage{bibentry}
\nobibliography*

% --- Table of Content
\setcounter{secnumdepth}{2}
\setcounter{tocdepth}{1}
\usepackage{minitoc}
\setcounter{minitocdepth}{1}

\newcommand\Chapter[2]{
	\chapter[#1 {#2}]{#1\\[0.8ex]\Large{#2}}
}

% --- Source code input, see https://en.wikibooks.org/wiki/LaTeX/Source_Code_Listings
\usepackage{framed}

% --- To Do notes
\ifsetDraft
	\usepackage[colorinlistoftodos]{todonotes}
	\newcommand{\TODO}[1]{\todo[inline]{TODO: #1}}
	\newcommand{\TODOE}[1]{\todo[inline,color=blue!40]{Edouard: #1}}
	\newcommand{\FIXME}[1]{\todo[inline]{FIXME: #1}}
\else
	\newcommand{\mynote}[1]{}
	\newcommand{\listoftodos}{}
	\newcommand{\TODO}[1]{}
	\newcommand{\TODOE}[1]{}
	\newcommand{\FIXME}[1]{}
\fi

% --- Remove some warnings ---
%\pdfoptionpdfminorversion=7
\usepackage{silence}
\pdfsuppresswarningpagegroup=1
\WarningFilter{minitoc(hints)}{W0023}
\WarningFilter{minitoc(hints)}{W0028}
\WarningFilter{minitoc(hints)}{W0030}

%% Theorem English
\newtheorem{theorem}{Theorem}[chapter]
\newtheorem{assumption}[theorem]{Assumption}
\newtheorem{claim}[theorem]{Claim}
\newtheorem{corollary}[theorem]{Corollary}
\newtheorem{definition}[theorem]{Definition}
\newtheorem{defn}[theorem]{Definition}
\newtheorem{example}[theorem]{Example}
\newtheorem{lemma}[theorem]{Lemma}
\newtheorem{notation}[theorem]{Notation}
\newtheorem{proposition}[theorem]{Proposition}
\newtheorem{remark}[theorem]{Remark}
\crefname{assumption}{assumption}{assumptions}


\makeatletter
% https://tex.stackexchange.com/a/365249/97964
\def\cleardoublepage{%
	\clearpage%
	\if@twoside%
		\ifodd\c@page%
		\else%
			\hbox{}\thispagestyle{empty}\newpage%
			\if@twocolumn%
				\hbox{}\newpage%
			\fi%
		\fi%
	\fi%
}
% https://tex.stackexchange.com/a/365249/97964
\newcommand*{\cleartoleftpage}{%
	\clearpage%
	\if@twoside%
		\ifodd\c@page%
			\hbox{}\thispagestyle{empty}\newpage%
			\if@twocolumn%
				\hbox{}\newpage%
			\fi%
		\fi%
	\fi%
}
\makeatother

% Macros from Émilie Kaufmann's articles
\usepackage{0-Misc/macrosText}

% FIXME can I use the font I like?
% https://tex.stackexchange.com/questions/84770/using-palatino-and-euler-math#comment182397_84770
\usepackage{tgpagella}
% \usepackage{eulervm}

\pagecolor{white}
\usepackage{pdfpages}

