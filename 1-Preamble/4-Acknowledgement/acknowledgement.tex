% ************************** Thesis Acknowledgements **************************

\begin{acknowledgements}
%A core concept in theoretical Reinforcement Learning is that of \emph{credit assignment}: upon reception of a high reward, 
Un concept central en apprentissage par renforcement est celui du \emph{«~credit assignment~»} («~attribution du mérite~»). Selon ce principe, lors de l'obtention d'une haute récompense, il convient de remonter l'historique des évenements survenus par le passé afin d'identifier ceux qui furent responsables de ce succès.
Prêtons-nous à l'exercice.

Tout d'abord, je dois tout à mes parents, ainsi qu'a mes soeurs et mon frère. Leur affection et leur soutien constants au fil des années m'ont permis de m'engager avec confiance dans cette aventure.

Mais je n'aurais pas entrepris cette thèse sans l'exemple éclatant des doctorants de Parrot, Gauthier Rousseau et Clément Pinard, ni le concours de Jill-Jênn Vie, Edouard Oyallon, Alberto Bietti et Michal Valko qui ont conspiré ensemble à me conduire au laboratoire SequeL. Je remercie également l'examen du permis de conduire, auquel mes échecs répétés m'ont permis de développer un intérêt égoiste mais pragmatique pour ce sujet de thèse en particulier.

J'adresse maintenant mes plus chaleureux remerciements à mes encadrants. Odalric et Denis, j'admire profondément votre intégrité et votre rigueur scientifique, ainsi que l'étendue de vos connaissances, que vous savez mobiliser pour répondre à la moindre de mes interrogations avec une facilité déconcertante. Merci également pour votre ouverture d'esprit, dont témoigne particulièrement la rencontre de vos disciplines respectives et la confrontation fructueuse des points de vues et des méthodes qui en découle. Mais avant tout, je vous suis reconnaissant pour votre bienveillance, votre disponibilité et votre soutien indéfectible lorsque j'en avais besoin.
Yann, je te remercie d'avoir monté ce projet ambitieux, et de m'avoir accordé une pleine liberté dans mes recherches, bien qu'elles se soient parfois écartées des préoccupations plus concrètes de l'ingénierie Renault. Enfin, je remercie Wilfrid pour ses précieux conseils toujours pertinents.

J'en viens à mes coreligionaires de la pause café, avec qui j'ai pu partager mes joies et mes intérêts, mes préoccupations et mes angoisses~; je leur dois mon équilibre mental. A SequeL tout d'abord, je remercie particulièrement Mathieu et Xuedong, camarades de la première heure avec qui j'ai partagé d'inoubliables marches aléatoires dans les montagnes de Stellenbosch~; Nicolas et Omar, avec qui il est plus que plaisant de collaborer~; 
Nathan et Dorian, dont l'appetance pour le débat n'a d'égale que leur propension à les trancher à coup d'études ad-hoc et de mise en équation~;  Lilian et Guillaume dont l'éloignement rendait la visite occasionelle d'autant plus festive~; Pierre M. dont la cinéphilie et l'hospitalité ont permis la renaissance en grande pompe du Ciné-Sequel~; Ronan, Merwan, Masha, Yannis, Reda, Romain, Pierre S., et Sarah. 

Enfin Renault, Jean, j'ai su dès notre rencontre à Munich que le goût communicatif pour la philosophie nous conduirait à d'interminables conversations. 
Merci également à Clara, Edwin, Federico, Lu, Louis et Thomas pour nos repas-mini-doc

CAOR, Philip, Florent et Marin.

Amis Adrien, Bertrand, Luc et Pierre.

Enfin, Ariane

\end{acknowledgements}

